\chapter{RESULTS AND DISCUSSION}



\section{I.V. v. degree}

As stated in the objectives serveral SOMs are to be trained with the same
training data and the same parameters.  The first step is to justify the
parameters I choose, and the training data I use.

The data will be genorated with the common technique of rejection sampeling,
where n high dimmensional seeds are choosen at random, samples will be taken
at random from a uniform distrobution, if the sample falls with in radius r of
any seed it is accepted, if it does not, single random number is drawn, if that
number is < 0.05\% the sample is accepted as noise, else the sample is rejected.

Currently implemented are the rectangular and sphereical topologies, the hexagon
and geodesic are to be implmented only as graphs.  I.E. the topologies will be
generated externally and turned into a networkX graph, for use in the graph
based SOM. The spherical topology also relies on extranal programs to generate
the graph structure (spherical voronoi).

How do we decide which parameters to use?

The number of dimensions is largely an arbitrary decision no?
5dims is easier to work with

show \ref{ivtable1}
show \ref{ivtable2}

\begin{table}
\caption{Mean Internal Variane for the entire som GRAPH}
\label{ivtable1}
\begin{tabular}{|c||c|c|c|c|}
\hline
&\multicolumn{4}{c|}{\textbf{Dimmensions}}\\
\textbf{Clusters} & \multicolumn{1}{c}{\textbf{2}} &
\multicolumn{1}{c}{\textbf{5}} & \multicolumn{1}{c}{\textbf{10}} &
\multicolumn{1}{c|}{\textbf{20}}\\
\hline
\hline
\textbf{0} & 0.0207& 0.2661& 0.7131& 1.3997 \\
\hline
\textbf{2} & 0.0106& 0.1106& 0.2550& 0.4872 \\
\hline
\textbf{5} & 0.0117& 0.0968& 0.2194& 0.4557 \\
\hline
\textbf{10} & 0.0118& 0.0839& 0.2051& 0.4174 \\
\hline
\textbf{20} & 0.0123& 0.0844& 0.1989& 0.4017 \\
\hline
\end{tabular} \end{table}



\begin{table}
\caption{Mean Internal Variane for the entire som ROOK}
\label{ivtable2}
\begin{tabular}{|c||c|c|c|c|}
\hline
&\multicolumn{4}{c|}{\textbf{Dimmensions}}\\
\textbf{Clusters} & \multicolumn{1}{c}{\textbf{2}} &
\multicolumn{1}{c}{\textbf{5}} & \multicolumn{1}{c}{\textbf{10}} &
\multicolumn{1}{c|}{\textbf{20}}\\
\hline
\hline
\textbf{0} & 0.0206& 0.2796& 0.7300& 1.4173 \\
\hline
\textbf{2} & 0.0114& 0.1140& 0.2598& 0.4930 \\
\hline
\textbf{5} & 0.0116& 0.0989& 0.2213& 0.4623 \\
\hline
\textbf{10} & 0.0117& 0.0873& 0.2071& 0.4207 \\
\hline
\textbf{20} & 0.0120& 0.0851& 0.2020& 0.4059 \\
\hline
\end{tabular} \end{table}



ok after adding these tables we find that 5dims and 10clusters is a reasonable
choice. 5 dims are easier to work with and yeild more information the 2dims.
After running 10 simulations for both the rook and grpah case we find that the
IV seems to remain fairly stable, this suggests that we can combine these to
compare accross topologies.
\ref{ivtable3}

\begin{table}
\caption{Mean IV for test case, graph vs. rook}
\label{ivtable3}
\begin{tabular}{|c||c|c|c|c|}
\hline
\textbf{Test Number} & Graph & Rook \\
\hline
\hline
\textbf{0} & 0.0910 & 0.0920 \\
\hline
\textbf{1} & 0.0874 & 0.0886 \\
\hline
\textbf{2} & 0.0809 & 0.0814 \\
\hline
\textbf{3} & 0.0812 & 0.0823 \\
\hline
\textbf{4} & 0.0907 & 0.0929 \\
\hline
\textbf{5} & 0.0938 & 0.0944 \\
\hline
\textbf{6} & 0.0908 & 0.0929 \\
\hline
\textbf{7} & 0.0792 & 0.0797 \\
\hline
\textbf{8} & 0.0974 & 0.1001 \\
\hline
\textbf{9} & 0.0897 & 0.0910 \\
\hline
\end{tabular} \end{table}
