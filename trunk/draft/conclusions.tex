%\chapter{Significance and Limitations}
\chapter{CONCLUSIONS}
The spherical SOM has been widely suggested as a solution to the edge effect
in traditional SOM.  The edge of the traditional SOM is a problem because neurons on
the edge of the topology are less central to the neural network than those in
the center of the topology, resulting is differing levels of influence on the
training process.  We extended this problem to spherical SOM by
asserting that irregular spherical topologies will have neurons that are less
central than others.  Existing research in spherical SOM has widely been
focused on minimizing this irregularity and the commonly used tessellated
icosahedron based topology offers the most regular topology. However, the main
disadvantage of this topology type is that it offers very limited control over
network size.  Alternative methods for generating the spherical topology,
which can create a network of any size, have been reviewed or suggested by
\cite{wu2005} and \cite{Nishio:2006fk}.  These alternative methods have been
largely dismissed because they produce network structures that are more irregular.
This research has taken a closer look at the impact that irregularity has on
the training process in an attempt to address the suitability of these less regular
topologies for use in SOM.

The objective of this research was to determine the utility of certain
irregular spherical topologies that offer greater control over
network size.  Toward that end, new diagnostic measures were developed that
allow for SOM comparisons based on topology-induced errors.  The diagnostics
measure the internal heterogeneity of observations captured by a given neuron
relative to that neuron's first-order neighborhood size, as well as relative
to a composite measure of topological regularity.

In this study the new diagnostics were applied to the evaluation of both
traditional and spherical SOMs. Each SOM was trained using the same synthetic
data and training parameters, but utilizing different network topologies.  By
formally testing for difference of means and variance in the results of the
diagnostics, the following conclusions were made: 1) The internal
heterogeneity of a neuron decreases as its first-order neighborhood size
increases, but less so in spherical topologies.  2) The average internal
heterogeneity of a SOM is higher when a more irregular topology is used.  3)
Visualizing internal heterogeneity provides valuable insight into the
underlying training process and how the SOM represents clusters. From these
conclusions, we can make interpretations on the importance of topological
irregularities in SOM.

In comparison to the tradition rectangular and hexagonal topologies, the effects
of irregular topology were minimized in both of the spherical
topologies we tested.  Specifically, we found no significant differences
between the two different spherical topologies we test.  As such, we believe
that the importance of regularity within spherical SOM may be overstated.
Relying only on highly regular spherical topologies may place
unnecessary constraints on the users of spherical SOM.  

In pursuit of these findings we built an open source implementation of SOM,
PySOM. This software is capable of training a SOM with an arbitrary
topology.  This allowed us to test the four different topologies presented in
the previous chapters.  Using PySOM, future researchers may train their
SOMs on a variety of different topologies.  We hope that this will enable an
exploratory based approach to help determine the most appropriate topology for
a given research question.

Traditionally the geographical sciences have been concerned with identifying and
understanding relationships found in the context of spatial data.  
The existing toolbox for studying these relationships is sizable and
and steadily growing as geographers develop new ways to look at spatial
information.  While the SOM and spherical SOM are a welcome addition to that
toolbox these tools provide much more than a new visualization method or
clustering technique.  Using SOM we can extract spatial representations from
otherwise aspatial data, allowing us to leverage our existing set of
tools on a whole new set of problems.

\section{Limitations and Future Directions}
A key limitation of this research is that we only address how topological
regularity effects the SOM.  We do not address how other attributes of the
topology effect the SOM. Specifically, the benefits and/or costs of removing the
edge are unclear.  More research needs to be done in order to determine if spherical SOM
is appropriate for all applications, or under which conditions it becomes
inappropriate.  The edge effect may in fact be useful in identifying outliers.




%\section{Significance}
%The commonly used tessellated icosahedron based topology offers the most regular topology.   However, the main disadvantage of this topology type is that it offers a limited control over network size.  Alternative methods for generating the spherical topology, which can create a network of any size, have been reviewed or suggested by \cite{wu2005} and \cite{Nishio:2006fk}.  These alternative methods produce network structures that are more irregular.  This research will take a closer look at the impact that the irregularity has on the training process in an attempt to address the suitability of these topologies for use in SOM.

%\section{Limitations}
%This research will look at the relationship between regularity in neuron connectedness and the training of a SOM. The relationship between topology and SOM visualization is not addressed. The topology chosen for a SOM has a direct link with how that SOM is visualized.  When representing the topology on the surface of a sphere issues arise with the uniformity in neuron spacing and sizing.  Future work may be needed to address these issues in visualization.


%\subsubsection{Train SOM}

%It is useful to represent our neural lattice as a graph, G.
%We know that any arrangement of neurons onto the surface of a sphere will as such the network can be effectively treated as a graph.  Doing so enables us to use the properties of the graph and the principles of graph theory to help us understand the relationship between neurons during training.

%the variance in neuron spacing should be minimized.    The first condition ensures that each neuron will occupy is only of concern when visualizing the SOM.

%The get at the second condition we must first define the degree of a neuron, deg(n) to be number of its direct neighbors.  Whit that in mind the second condition is that the variance in the degree of the neurons must also be minimized.

%This statement may be somewhat misleading to those investigating alternative 

%For the purpose of SOM visualization it is important for each neuron to receive equal geometrical treatment.

%It is useful to represent the neural network as a graph, G, in order use the...
%In order to examine the irregularities in the neural network it is useful ....

%The basic problem of the ``boundary effect'' is that neurons on the edge have fewer neighbors. Yet there are only five possible arrangements of points on a sphere such that all points have the same number of neighbors.  Any spherical lattice consisting of more then twenty (dodecahedron) neurons will contain topological irregularities.  This is to say that not all neurons will have the same number of neighbors.  The importance of these irregularities and the magnitude of their effects on SOM training is not known.  This goal of this research is to determine whether more flexible network structures may be used in spherical in SOM without introducing significant errors. To accomplish this goal, basic methods in network analysis with be combined with the result from several empirical training runs each utilized different topology.  
