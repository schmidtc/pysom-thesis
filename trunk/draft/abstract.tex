The traditional Self-Organizing Map (SOM) uses a rectangular or hexagonal
network topology.  These topologies create a well-known problem in SOM called
the boundary or edge effect.  Toroidal and spherical topologies have been widely
suggested as a solution to the problem.  However, there exist only five
arrangements on the sphere which are completely regular; these are the five
platonic solids (Ritter, 1999; Harris et al., 2000).  Any other arrangement of
neurons on the surface of the sphere will result in an irregular topology, as
not all neurons will have the same number of neighbors. The classic method for
minimizing this irregularity is to generate the spherical lattice by
tessellating the sides of the icosahedron (Nishio et al., 2006). While this
method will always result in a highly regular spherical topology, the main
drawback is that the number of neurons in the network (the network size), N,
grows exponentially as tessellations are applied. That results in only very
coarse control over network size.  We explore the effect of different topologies
on properties of SOM. We suggest several diagnostics for measuring
topology-induced errors in SOM and use these in a comparison of four different
topologies. The results show that SOM is less sensitive to localized
irregularities in the network structure than the literature may otherwise
suggest.  Further, the results support the use of spherical topologies as a
solution to the boundary problem in traditional SOM.
