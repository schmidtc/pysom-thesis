
%
%  $Description: Author guidelines and sample document in LaTeX 2.09$ 
%
%  $Author: ienne $
%  $Date: 1995/09/15 15:20:59 $
%  $Revision: 1.4 $
%

\documentclass[times, 10pt,twocolumn]{article} 
\usepackage{latex8}
\usepackage{times}

\usepackage{graphicx} %sjr added
\graphicspath{{../presentation/figures/}{../draft/figures/}}
%\documentstyle[times,art10,twocolumn,latex8]{article}

%------------------------------------------------------------------------- 
% take the % away on next line to produce the final camera-ready version 
\pagestyle{empty}

%------------------------------------------------------------------------- 
\begin{document}

\title{Effects of Irregular Topology in Spherical Self-Organizing Maps}

\author{
Charles R. Schmidt\\
School of Geographical Sciences\\Arizona State University\\
Charles.R.Schmidt@asu.edu\and
Sergio J. Rey\\
School of Geographical Sciences\\Arizona State University\\srey@asu.edu
\and
Andr\'e Skupin\\
Department of Geography\\San Diego State University\\
skupin@mail.sdsu.edu
}

\maketitle
\thispagestyle{empty}

\begin{abstract}
We explore the effect of different topologies on properties of self-organizing
maps (SOM). We suggest several diagnostics for measuring
topology-induced errors in SOM and use these in a comparison of four different
topologies. The results $\ldots$
  
\end{abstract}



%------------------------------------------------------------------------- 
\Section{Introduction}

The Self-Organizing Map (SOM) is an unsupervised competitive learning process
developed by Teuvo Kohonen as a technique to analyze and visualize high
dimensional data sets.  The applications of SOM are far reaching;
Kohonen \cite{Kohonen2000} provides a thorough review of the SOM literature including
applications of SOM.  SOM has been used in applications ranging from speech
recognition and image classification to breast cancer detection and gene
expression clustering.  Agarwal
and Skupin \cite{skupin07} outline the growing interest of SOM to
the GISciences, and propose that the relationship between SOM and GIScience
should be bidirectional.  The SOM offers a powerful method for exploring and
visualizing geographic data and GIScience offers a wide array of tools
and methods to enable the exploration of the SOM itself.  The exploration of
spatial relationships has always been of great interest to geographers, and as
Ritter
\cite{ritter99} states, the goal of SOM is ``to translate \emph{data
similarities} into \emph{spatial relationships}'' \cite[p. 1]{ritter99}. 

The SOM is a type of artificial neural network in which neurons are ``organized''
in such a way as to project the high-dimensional relationships of a set of
training data onto a low-dimensional network structure.  The traditional
SOM uses a rectangular or hexagonal network topology \cite{Kohonen2000}.  These topologies 
create a well-known problem in SOM called the boundary or edge effect.  Neurons on
the boundary of the hexagonal and rectangular lattices have fewer neighbors,
which reduces their ability to interact with other neurons during the
self-organizing process.  Using a spherical lattice has been widely suggested as a
solution to the problem \cite{ritter99, boudjemai2003, sangole03,
Nishio:2006fk, wu2006}. The use of the spherical lattice, however, does not
completely overcome the boundary problem, and the choice of which spherical
topology to use for the network can be difficult to make.

A regular network topology is one in which every node on the network has exactly the
same number of adjacent nodes.  Any topology involving an edge is irregular.
Arranging our lattice on the surface of a sphere seems to be an obvious
way to overcome the edge.  However, there exist only five arrangements on the
sphere which are completely regular; these are the five platonic solids \cite{ritter99,
harris2000}.  Any other arrangement of neurons on the surface of the sphere will
result in an irregular topology, as not all neurons will have the same number of
neighbors.

The classic method for minimizing this irregularity is to generate
the spherical lattice by tessellating (subdividing) the sides of the icosahedron
\cite{Nishio:2006fk}.  While this method will always result in a highly
regular spherical topology, the main drawback is that the number of neurons in
the network (the network size), \(N\), grows exponentially as tessellations are
applied. That results in only very coarse control over network size.
 Other methods for arranging neurons on the sphere allow
for unlimited control over network size, but yield topologies with increased
irregularity \cite{harris2000, wu2005, Nishio:2006fk}.  To date the
literature has largely ignored the more irregular methods in favor of the
aforementioned tessellation-based methods.  A topology which yields a more flexible network
size may be desirable.  However, in order to address this issue of network
size, we must first determine the degree to which irregularity effects the
SOM.

\Section{Edge Effects}

An intrinsic problem with SOMs in two dimensions is the so called edge
effect. This is illustrated in  in Figure \ref{f:edge}, which displays a SOM
trained on socioeconomic census data consisting of 32 variables for US states.
The darker a neuron in the figure, the larger is the distance between the
 neuron and mean of the input vectors. The distance between each of the
 original observations (states) and the mean is represented by a graduated
 circle. Taken these two together it is clear that the outlier observations
 are pushed towards the edges of the map, while the observations that are
 closest to the multidimensional center of the data are assigned to central
 neurons on the map.

At the edge of the map, neurons have fewer neighbors which results in any
observations being assigned to them having fewer competitive signals. At the
same time, the edge of the neural lattice represents a true visual boundary
which affects its ability to represent data similarities as spatial
relationships.

\begin{figure}
  \begin{center}
\caption{Edge effects in SOM}\label{f:edge}
  \includegraphics[width=0.70\linewidth]{states.png}
\end{center}
\end{figure}

One way to eliminate the edge effect is to wrap the lattice around a
three-dimensional object such as a sphere or torus, thereby removing the edge
entirely. The toroidal SOM was introduced by \cite{li1993}, however the torus
is not effective for visualization, as maps generated from a torus are not
very intuitive \cite{ito2000,wu2006}.  \cite{ritter99} describes the torus as
being topologically flat and suggests that a curved topology, such as that of
a sphere, may better reflect directional data.  A sphere also results in a
more intuitive map, since we are accustomed to looking at geographic maps
based on a sphere.  

\label{bg:sphere}
Ritter \cite{ritter99} first introduced the spherical SOM, and several enhancements have
since been suggested \cite{boudjemai2003,sangole03,Nishio:2006fk,wu2006}.  A
good comparison of these enhancements can be found in Wu and Takatsuka
 \cite{wu2006}.  All of
these methods derive their spherical structure through the tessellation of a
polyhedron as originally proposed by Ritter \cite{ritter99}.  Wu and Takatsuka \cite{wu2006} point
out the importance of a uniform distribution on the sphere, and that it is
preferable for all neurons to have an equal number of neighbors and to be
equally spaced.  They find generally that the tessellation method best satisfies
these conditions, and specifically that the icosahedron is the best starting
point \cite{wu2005}. Tessellation of the icosahedron results in a network of
neurons, each having exactly six neighbors, save the original twelve
which each have five neighbors.  This is very close to the ideal structure in
which every neuron would have exactly six neighbors.  \cite{wu2006} prefer
these this structure, because it has very low variances in both neuron spacing
and neighborhood size. 

Based solely on measures of neuron spacing, \cite{wu2005} dismissed the usefulness of a method
proposed by \cite{Rakhmanov94} for distributing points on a sphere.  Similarly
\cite{Nishio:2006fk} use these variance measures to support their helix
algorithm for distributing points on a sphere.  However,
these metrics can be misleading and comparison across topologies may not be
consistent.  The traditional rectangular and hexagonal topologies have no
variance in neuron spacing, and the generally preferred hexagonal structure
displays greater variance in neighborhood size than the rectangular structure.
The torus, by comparison, would have variance in neuron spacing, yet no
variance in neighborhood size.  The distance between two neurons is only
considered during the formation of the neural network.  At this stage the
spacing is significant as it plays a part in constructing the network's
topology by determining neuron adjacency.  However, using this measure to
evaluate potential topologies for use in SOM may be misleading.

As spherical (and other alternative) topologies become
increasingly more common it is necessary to investigate how the choice of
topology effects the SOM.  In this thesis the effect of irregularity within
topologies is studied as an attempt to investigate not only the edge effect,
but also to help facilitate the comparison of topologies.  It is important to
note that spherical topologies may not be appropriate for all applications.
Removing the edge may reduce the SOM's ability to converge.  As outliers are
forced to interact they introduce more competition among the neurons.  We
would also expect outliers to occupy more space in the final map as their
dissimilarity in attribute space should translate to more distant spatial
relationships in the trained SOM.  More research will be needed to help researches
determine the most appropriate topology for their data and research objectives.

In this paper we explore the general utility of certain irregular spherical
topologies beyond offering greater control over network size. We develop and
test new diagnostics to measure and visualize topology-induced errors in SOM.
More specifically we examine the



\Section{Methods}
\chapter{METHODOLOGY}
This chapter is composed of two sections.  The first describes the diagnostics
developed for evaluating network topologies.  The second describes an
empirical study that will implement these methods in order to evaluate the
utility of topologies that allow for greater control over network size.

\section{Diagnostics}
Three diagnostics are developed to explore the effect of irregular topology on
spherical SOM.  The first diagnostic will be used to address the research
question regarding the internal variance and neighborhood size.  The second
diagnostic will address the question concerning internal variance and
topological irregularity.  The third tool will help visualize the patterns
between internal variance and topology.

\subsection{Internal variance vs. first-order neighborhood size}
\label{q1}
This diagnostic will compare the internal variance of each neuron against its
first-order neighborhood size.  In traditional SOMs, outlying observations are
pushed to the edge of the map where they encounter fewer competing signals. A
prime example of this is the ``Utah-Hawaii'' case shown in Figure
\ref{figure1}.  Relying only on the SOM, one would be left to believe that the
two states are similar. Upon closer inspection we see that the QError from
Utah to the neuron is $1.509$, the QError from Hawaii to the neuron in
$1.505$, but the QError from Utah to Hawaii is $3.014$. In this case only Utah
and Hawaii were mapped to that neuron.  In a case where multiple observations
land on the same neuron, it is possible to measure average pairwise QErrors
between those observations.  This gives us a notion of internal variance for
each neuron. It would be expected that in traditional SOMs neurons closer to
the edge will have higher internal variances. This can be extended to
spherical SOMs by comparing the degree of a neuron ($deg(m_i)$ or the number of
adjacent neurons) to its internal variance.  The degree of each neuron can
easily be calculated by taking the column sums of the first-order adjacency
matrix ($A$).

Once the internal variance ($var(m_i)$) and degree ($deg(m_i)$) of each neuron
have been calculated, the neurons can be separated into a small number of
groups based on the degree \footnote{For most topologies the number of
different degrees will be limited to three or four.}.  The variance and mean
will be calculated for each of these groups.  The expected result is that
variances and means of the groups will decrease as the degree increases.  This
hypothesis will be tested using random labeling as described by \cite{siss2004}.
The result will also be visualized using a box plot.

\subsection{Internal variance vs. centrality}
The degree of a node on a network is a measure of its centrality, or
importance. Nodes with more connections are thought to be more central to
network and have a larger influence than nodes with fewer connections. As an
artifact of the training process observations that are more average than
others tend to be centralized.  The observations that sound them tend to be
more extreme.  If you refer back to figure \ref{figure1}, you'll notice that
observations with smaller symbols are closer to the mean of all the
observations and that these observations have been centralized in the network.

Using the degree as a measure of centrality does not capture this picture
well, as neurons near the edge can still have a large degree.  A way to
capture this effect would be to look at closeness centrality.  This is, the
inverse of the average distance of a neuron to every other neuron on the
network.


This diagnostic will compare the internal variance of each neuron against a
summary measure of centrality for its associated topology. Centrality measures
the importance of a node in the graph.  The most central node is 

As mentioned above the
degree of each neuron can be calculated by taking the column sums of $A$.  A
completely regular network topology (i.e. the torus) will have no variance
between these column sums.  For irregular networks the variance between these
column sums gives us a measure of irregularity.  An alternative to using a
measure of regularity would be a measure of connectivity or centrality.  
\cite{florax95} outline four measures of connectivity and analyze their
sensitivity to network size.
%There are many ways to classify the connectivity of a network; such summary measures.
For each topology we can compare the internal variances as described above
against a measure that summarizes the given topology's regularity.

This diagnostic is evaluated in much the same way as the last diagnostic.  
The internal variances are this time grouped by their topology.  We can then
compare the variances of internal variances and the means of the internal
variances across topologies.  It is expected that the distribution of internal
variances will be narrower for groups trained on more regular topologies.
It is further hypothesized that the means of these internal variances will
decrease when the network is more regular, or when there is less variance in
the column sums of $A$.  These assumptions will be tested using
a \emph{t-test} on the means and an \emph{F-test} on the variances.

\subsection{Visualize internal variance mapping}
Visualizing the internal variance may yield insight into how irregular topology
effects the SOM.  Once the internal variance of each neuron has been calculated
we can use the values to color or shade a map of the given topology.  The degree
of the neurons can be visualized using proportional symbols to help show
patterns between internal variance and irregularity.

\section{Empirical Analysis}
The empirical analysis consists of three main tasks.  The first is to create
synthetic data suitable for the diagnostics described above.  The second task is
to train multiple SOMs, each of with a different topology type. The third task
will be to apply the diagnostics and interpret the results.

\subsection{Synthetic Data}
%Comment from Skupin...
%This section is obviously leaving most of the specifics of the synthetic data
%generation out, which is problematic. I'm willing to go along with this for
%the proposal though, unless it gets raised by the third committee member
In order to test the methods described above, I will generate high-dimensional
synthetic data with known properties.  Knowing the properties of the training
data allow us to systematically compare the diagnostics under several different
topologies.  To ensure that we can calculate an internal variance for each
neuron, I will generate the data so as to increase the probability that each
neuron will be occupied by more than one observation.

The data will be generated using a Gaussian cluster generator as described by
\cite{handl}. The generator creates clusters by pulling from multivariate
normal distributions.  Clusters are not allowed to overlap and no random noise
is introduced into the data.  The properties such as the number of clusters
and the number of dimensions will be decided through experimentation.  After
the data is generated, it is scaled from zero to one and the order of
randomized.  We will create a number of different data sets and use them to
train various SOMs.  The mean internal variance of each SOM will be looked at
to determine how it responds to the properties of the training data.

\subsection{Training}
The diagnostics must be given a trained SOM for each topology to be compared. To
yield any meaningful results those SOMs must be trained with comparable
parameters. Most parameters can simply be set to the same value for each SOM.
However, special consideration must be given to network size.  As shown in
Figure \ref{fig:nSize}, topologies differ in terms of achievable network size.
This analysis will include the following topologies:
\begin{itemize}
\item Rectangular
% Comment by Skupin note addressed here.  Need to come up with better
% name...
\item Hexagonal
\item A topology based on \cite{Rakhmanov94}
\item The Helix topology proposed by \cite{Nishio:2006fk}\footnote{Currently
there is some uncertainty about the ability to include the Geodesic and Helix
type topologies given the complexities involved with their implementation.
However, an additional goal of this project to provide a framework on which new
topologies can be easily implemented and tested by future researchers.}
\item The Geodesic topology proposed by \cite{wu2006}\footnotemark[2]
\end{itemize}

\subsection{Diagnostics}
The first diagnostic will yield a set of results for each topology tested.  These
results will be analyzed in order to address the first research question of this
paper.  The second diagnostic provides one set of results.  These results will
be analyzed to address the second research question.  The final diagnostic will
return a visualization for each topology tested.  The usefulness of these
visualizations is a research question in itself.  The expectation is that the
visualizations will show patterns of internal variance related to irregularities
in the network topology.


%------------------------------------------------------------------------- 

%------------------------------------------------------------------------- 
\SubSection{Language}

All manuscripts must be in English.

%------------------------------------------------------------------------- 
\SubSection{Printing your paper}

Print your properly formatted text on high-quality, $8.5 \times 11$-inch 
white printer paper. A4 paper is also acceptable, but please leave the 
extra 0.5 inch (1.27 cm) at the BOTTOM of the page.

%------------------------------------------------------------------------- 
\SubSection{Margins and page numbering}

All printed material, including text, illustrations, and charts, must be 
kept within a print area 6-7/8 inches (17.5 cm) wide by 8-7/8 inches 
(22.54 cm) high. Do not write or print anything outside the print area. 
Number your pages lightly, in pencil, on the upper right-hand corners of 
the BACKS of the pages (for example, 1/10, 2/10, or 1 of 10, 2 of 10, and 
so forth). Please do not write on the fronts of the pages, nor on the 
lower halves of the backs of the pages.


%------------------------------------------------------------------------ 
\SubSection{Formatting your paper}

All text must be in a two-column format. The total allowable width of 
the text area is 6-7/8 inches (17.5 cm) wide by 8-7/8 inches (22.54 cm) 
high. Columns are to be 3-1/4 inches (8.25 cm) wide, with a 5/16 inch 
(0.8 cm) space between them. The main title (on the first page) should 
begin 1.0 inch (2.54 cm) from the top edge of the page. The second and 
following pages should begin 1.0 inch (2.54 cm) from the top edge. On 
all pages, the bottom margin should be 1-1/8 inches (2.86 cm) from the 
bottom edge of the page for $8.5 \times 11$-inch paper; for A4 paper, 
approximately 1-5/8 inches (4.13 cm) from the bottom edge of the page.

%------------------------------------------------------------------------- 
\SubSection{Type-style and fonts}

Wherever Times is specified, Times Roman may also be used. If neither is 
available on your word processor, please use the font closest in 
appearance to Times that you have access to.

MAIN TITLE. Center the title 1-3/8 inches (3.49 cm) from the top edge of 
the first page. The title should be in Times 14-point, boldface type. 
Capitalize the first letter of nouns, pronouns, verbs, adjectives, and 
adverbs; do not capitalize articles, coordinate conjunctions, or 
prepositions (unless the title begins with such a word). Leave two blank 
lines after the title.

AUTHOR NAME(s) and AFFILIATION(s) are to be centered beneath the title 
and printed in Times 12-point, non-boldface type. This information is to 
be followed by two blank lines.

The ABSTRACT and MAIN TEXT are to be in a two-column format. 

MAIN TEXT. Type main text in 10-point Times, single-spaced. Do NOT use 
double-spacing. All paragraphs should be indented 1 pica (approx. 1/6 
inch or 0.422 cm). Make sure your text is fully justified---that is, 
flush left and flush right. Please do not place any additional blank 
lines between paragraphs. Figure and table captions should be 10-point 
Helvetica boldface type as in
\begin{figure}[h]
   \caption{Example of caption.}
\end{figure}

\noindent Long captions should be set as in 
\begin{figure}[h] 
   \caption{Example of long caption requiring more than one line. It is 
     not typed centered but aligned on both sides and indented with an 
     additional margin on both sides of 1~pica.}
\end{figure}

\noindent Callouts should be 9-point Helvetica, non-boldface type. 
Initially capitalize only the first word of section titles and first-, 
second-, and third-order headings.

FIRST-ORDER HEADINGS. (For example, {\large \bf 1. Introduction}) 
should be Times 12-point boldface, initially capitalized, flush left, 
with one blank line before, and one blank line after.

SECOND-ORDER HEADINGS. (For example, {\elvbf 1.1. Database elements}) 
should be Times 11-point boldface, initially capitalized, flush left, 
with one blank line before, and one after. If you require a third-order 
heading (we discourage it), use 10-point Times, boldface, initially 
capitalized, flush left, preceded by one blank line, followed by a period 
and your text on the same line.

%------------------------------------------------------------------------- 
\SubSection{Footnotes}

Please use footnotes sparingly%
\footnote
   {%
     Or, better still, try to avoid footnotes altogether.  To help your 
     readers, avoid using footnotes altogether and include necessary 
     peripheral observations in the text (within parentheses, if you 
     prefer, as in this sentence).
   }
and place them at the bottom of the column on the page on which they are 
referenced. Use Times 8-point type, single-spaced.


%------------------------------------------------------------------------- 
\SubSection{References}

List and number all bibliographical references in 9-point Times, 
single-spaced, at the end of your paper. When referenced in the text, 
enclose the citation number in square brackets, for example~\cite{ex1}. 
Where appropriate, include the name(s) of editors of referenced books.

%------------------------------------------------------------------------- 
\SubSection{Illustrations, graphs, and photographs}

All graphics should be centered. Your artwork must be in place in the 
article (preferably printed as part of the text rather than pasted up). 
If you are using photographs and are able to have halftones made at a 
print shop, use a 100- or 110-line screen. If you must use plain photos, 
they must be pasted onto your manuscript. Use rubber cement to affix the 
images in place. Black and white, clear, glossy-finish photos are 
preferable to color. Supply the best quality photographs and 
illustrations possible. Penciled lines and very fine lines do not 
reproduce well. Remember, the quality of the book cannot be better than 
the originals provided. Do NOT use tape on your pages!

%------------------------------------------------------------------------- 
\SubSection{Color}

The use of color on interior pages (that is, pages other
than the cover) is prohibitively expensive. We publish interior pages in 
color only when it is specifically requested and budgeted for by the 
conference organizers. DO NOT SUBMIT COLOR IMAGES IN YOUR 
PAPERS UNLESS SPECIFICALLY INSTRUCTED TO DO SO.

%------------------------------------------------------------------------- 
\SubSection{Symbols}

If your word processor or typewriter cannot produce Greek letters, 
mathematical symbols, or other graphical elements, please use 
pressure-sensitive (self-adhesive) rub-on symbols or letters (available 
in most stationery stores, art stores, or graphics shops).

%------------------------------------------------------------------------ 
\SubSection{Copyright forms}

You must include your signed IEEE copyright release form when you submit 
your finished paper. We MUST have this form before your paper can be 
published in the proceedings.

%------------------------------------------------------------------------- 
\SubSection{Conclusions}

Please direct any questions to the production editor in charge of these 
proceedings at the IEEE Computer Society Press: Phone (714) 821-8380, or 
Fax (714) 761-1784.

%------------------------------------------------------------------------- 
\nocite{ex1,ex2}
\bibliographystyle{latex8}
\bibliography{../proposal/som}

\end{document}

