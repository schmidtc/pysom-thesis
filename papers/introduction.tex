
The Self-Organizing Map (SOM) is an unsupervised competitive learning process
developed by Teuvo Kohonen as a technique to analyze and visualize high
dimensional data sets.  The applications of SOM are far reaching;
Kohonen \cite{Kohonen2000} provides a thorough review of the SOM literature including
applications of SOM.  SOM has been used in applications ranging from speech
recognition and image classification to breast cancer detection and gene
expression clustering.  Agarwal
and Skupin \cite{skupin08} outline the growing interest of SOM to
the GISciences, and propose that the relationship between SOM and GIScience
should be bidirectional.  The SOM offers a powerful method for exploring and
visualizing geographic data and GIScience offers a wide array of tools
and methods to enable the exploration of the SOM itself.  The exploration of
spatial relationships has always been of great interest to geographers, and as
Ritter
 states, the goal of SOM is ``to translate \emph{data
similarities} into \emph{spatial relationships}'' \cite[p. 1]{ritter99}. 

The SOM is a type of artificial neural network in which neurons are ``organized''
in such a way as to project the high-dimensional relationships of a set of
training data onto a low-dimensional network structure.  The traditional
SOM uses a rectangular or hexagonal network topology \cite{Kohonen2000}.  These topologies 
create a well-known problem in SOM called the boundary or edge effect.  Neurons on
the boundary of the hexagonal and rectangular lattices have fewer neighbors,
which reduces their ability to interact with other neurons during the
self-organizing process.  Using a spherical lattice has been widely suggested as a
solution to the problem \cite{ritter99, boudjemai2003, sangole03,
Nishio:2006fk, wu2006}. The use of the spherical lattice, however, does not
completely overcome the boundary problem, and the choice of which spherical
topology to use for the network can be difficult to make.

A regular network topology is one in which every node on the network has exactly the
same number of adjacent nodes.  Any topology involving an edge is irregular.
Arranging our lattice on the surface of a sphere seems to be an obvious
way to overcome the edge.  However, there exist only five arrangements on the
sphere which are completely regular; these are the five platonic solids \cite{ritter99,
harris2000}.  Any other arrangement of neurons on the surface of the sphere will
result in an irregular topology, as not all neurons will have the same number of
neighbors.

The classic method for minimizing this irregularity is to generate
the spherical lattice by tessellating (subdividing) the sides of the icosahedron
\cite{Nishio:2006fk}.  While this method will always result in a highly
regular spherical topology, the main drawback is that the number of neurons in
the network (the network size), \(N\), grows exponentially as tessellations are
applied. That results in only very coarse control over network size.
 Other methods for arranging neurons on the sphere allow
for unlimited control over network size, but yield topologies with increased
irregularity \cite{harris2000, wu2005, Nishio:2006fk}.  To date the
literature has largely ignored the more irregular methods in favor of the
aforementioned tessellation-based methods.  A topology which yields a more flexible network
size may be desirable.  However, in order to address this issue of network
size, we must first determine the degree to which irregularity effects the
SOM.


