\chapter{RESULTS AND DISCUSSION}




\section{I.V. v. degree}

As stated in the objectives several SOMs are to be trained with the same
training data and the same parameters.  The first step is to justify the
parameters I choose, and the training data I use.

The data will be generated with the common technique of rejection sampling,
where n high dimensional seeds are chosen at random, samples will be taken
at random from a uniform distribution, if the sample falls with in radius r of
any seed it is accepted, if it does not, single random number is drawn, if that
number is < 0.05\% the sample is accepted as noise, else the sample is rejected.

Currently implemented are the rectangular and spherical topologies, the hexagon
and geodesic are to be implemented only as graphs.  I.E. the topologies will be
generated externally and turned into a networkX graph, for use in the graph
based SOM. The spherical topology also relies on external programs to generate
the graph structure (spherical voronoi).

How do we decide which parameters to use?

The number of dimensions is largely an arbitrary decision no?
5dims is easier to work with

show \ref{ivtable1}
show \ref{ivtable2}

\begin{table}
\caption{Mean Internal Variane for the entire som GRAPH}
\label{ivtable1}
\begin{tabular}{|c||c|c|c|c|}
\hline
&\multicolumn{4}{c|}{\textbf{Dimmensions}}\\
\textbf{Clusters} & \multicolumn{1}{c}{\textbf{2}} &
\multicolumn{1}{c}{\textbf{5}} & \multicolumn{1}{c}{\textbf{10}} &
\multicolumn{1}{c|}{\textbf{20}}\\
\hline
\hline
\textbf{0} & 0.0207& 0.2661& 0.7131& 1.3997 \\
\hline
\textbf{2} & 0.0106& 0.1106& 0.2550& 0.4872 \\
\hline
\textbf{5} & 0.0117& 0.0968& 0.2194& 0.4557 \\
\hline
\textbf{10} & 0.0118& 0.0839& 0.2051& 0.4174 \\
\hline
\textbf{20} & 0.0123& 0.0844& 0.1989& 0.4017 \\
\hline
\end{tabular} \end{table}



\begin{table}
\caption{Mean Internal Variane for the entire som ROOK}
\label{ivtable2}
\begin{tabular}{|c||c|c|c|c|}
\hline
&\multicolumn{4}{c|}{\textbf{Dimmensions}}\\
\textbf{Clusters} & \multicolumn{1}{c}{\textbf{2}} &
\multicolumn{1}{c}{\textbf{5}} & \multicolumn{1}{c}{\textbf{10}} &
\multicolumn{1}{c|}{\textbf{20}}\\
\hline
\hline
\textbf{0} & 0.0206& 0.2796& 0.7300& 1.4173 \\
\hline
\textbf{2} & 0.0114& 0.1140& 0.2598& 0.4930 \\
\hline
\textbf{5} & 0.0116& 0.0989& 0.2213& 0.4623 \\
\hline
\textbf{10} & 0.0117& 0.0873& 0.2071& 0.4207 \\
\hline
\textbf{20} & 0.0120& 0.0851& 0.2020& 0.4059 \\
\hline
\end{tabular} \end{table}



ok after adding these tables we find that 5dims and 10clusters is a reasonable
choice. 5 dims are easier to work with and yield more information the 2dims.
After running 10 simulations for both the rook and graph case we find that the
IV seems to remain fairly stable, this suggests that we can combine these to
compare across topologies.
\ref{ivtable3}

\begin{table}
\caption{Mean IV for test case, graph vs. rook}
\label{ivtable3}
\begin{tabular}{|c||c|c|c|c|}
\hline
\textbf{Test Number} & Graph & Rook \\
\hline
\hline
\textbf{0} & 0.0910 & 0.0920 \\
\hline
\textbf{1} & 0.0874 & 0.0886 \\
\hline
\textbf{2} & 0.0809 & 0.0814 \\
\hline
\textbf{3} & 0.0812 & 0.0823 \\
\hline
\textbf{4} & 0.0907 & 0.0929 \\
\hline
\textbf{5} & 0.0938 & 0.0944 \\
\hline
\textbf{6} & 0.0908 & 0.0929 \\
\hline
\textbf{7} & 0.0792 & 0.0797 \\
\hline
\textbf{8} & 0.0974 & 0.1001 \\
\hline
\textbf{9} & 0.0897 & 0.0910 \\
\hline
\end{tabular} \end{table}


this table shoes the means ands variances for the degree groups, \ref{meanvar1}


\begin{table}
\caption{Mean \& (var) IV for degree groups in test cases, graph vs. rook}
\label{meanvar1}
\begin{tabular}{|l|c|c|}
\hline
\textbf{Degree Size} & \textbf{Graph} & \textbf{Rook} \\
\hline
\textbf{2}   (40) & & 0.1123 (0.00074)  \\
\textbf{3}  (918) & & 0.0997 (0.00095)  \\
\textbf{4} (5138) & & 0.0875 (0.00089)  \\
\textbf{5}  (557) &   0.0886 (0.00087) &\\
\textbf{6} (5034) &   0.0883 (0.00087) &\\
\textbf{7}  (437) &   0.0873 (0.00092) &\\
\hline
\end{tabular} \end{table}



here are the box plots for the IV. \ref{fRookIV} \ref{fGraphIV}


\begin{figure}
\centering
\includegraphics[width=\linewidth]{rook_iv_box.png}
\caption{This shows 3 box plots, each representing one group of neurons in a set
of SOMs trained with the same paremeters.}
\label{fRookIV}
\end{figure}

\begin{figure}
\centering
\includegraphics[width=\linewidth]{graph_iv_box.png}
\caption{This shows 3 box plots, each representing one group of neurons in a set
of SOMs trained with the same paremeters.}
\label{fGraphIV}
\end{figure}



\section{Restate the Questions}
\textbf{Objective}, Compare the internal variance of observations captured by a given
neuron to that neuron's first-order neighborhood size.

\textbf{Question}, Does the internal variance of a neuron decrease as its first-order
neighborhood size, or degree, increases?


To answer this question we need to setup a distance matrix (the distance of the
means) between each set of groups with in a given topology.  The table should
look like this...... 

  2 3 4
2 0 + +
3 - 0 +
4 - - 0

differance of means test\ldots
\ref{randomLabelTable}

\begin{table}
\caption{Random Labeling Mean Tests,  delta (p-Value)}
\label{randomLabelTable}
\begin{tabular}{|c||c|c|c|}
\hline
&2&3&4\\
\hline
\hline
2& & 0.012681 (0.006600)& 0.024790 (0.000100)\\
\hline
3& -0.012681 (0.991400)& & 0.012109 (0.000100)\\
\hline
4& -0.024790 (1.000000)& -0.012109 (1.000000)& \\
\hline
\end{tabular} \end{table}






