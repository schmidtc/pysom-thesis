REDO REDO REDO REDO REDO REDO REDO REDO REDO REDO REDO REDO REDO REDO REDO REDO
REDO REDO REDO REDO REDO REDO REDO REDO REDO REDO REDO REDO REDO REDO REDO REDO
REDO REDO REDO REDO REDO REDO REDO REDO REDO REDO REDO REDO REDO REDO REDO REDO
REDO REDO REDO REDO REDO REDO REDO REDO REDO REDO REDO 

The Self-Organizing Map (SOM) is a widely used technique in information
visualization and exploration.  The development of the spherical SOM has been
driven by the border effects observed in traditional SOM.  This thesis will
(1) investigate how irregular network topologies affect the SOM, (2) examine any
benefits spherical topologies offer over traditional planar topologies, and (3)
examine the trade off between increasingly regular topology versus greater
control over network size.  

REDO REDO REDO REDO REDO REDO REDO REDO REDO REDO REDO REDO REDO REDO REDO REDO
REDO REDO REDO REDO REDO REDO REDO REDO REDO REDO REDO REDO REDO REDO REDO REDO
REDO REDO REDO REDO REDO REDO REDO REDO REDO REDO REDO REDO REDO REDO REDO REDO
REDO REDO REDO REDO REDO REDO REDO REDO REDO REDO REDO 
