\documentclass[11pt]{article}
\begin{document}
\title{Distributed Spherical SOM in Python}
\author{\sc{Charles Schmidt}\\Regional Analysis Laboratory\\Department of Geography\\San Diego State University}

\date{October 10th, 2006}
\maketitle
\begin{abstract}
Traditionally self organizing maps make use a rectangular or hexagonal grid structure.  These traditional formats have pronounced edge effects that distort the data.  Spherical SOMs have been introduced in order to eliminate these edge effects.  However, evenly distributing points on a sphere is a difficult task with few exact solutions.  Rakhmonv E. A, et. al. (1994) provide a formula for distributing points on a sphere in a near even manner.  I suggest that using the N nearest neighbors to determine neighborhood size will reduce problems caused by non even point distribution.  Further more using nearest neighbors will allow for rapid development of a spherical SOM using existing tools and libraries.

Problems with GeoSOM\ldots
Wu and Takatsuka (2006) propose the use of a tessalted icosahedron to create a
geodesic SOM.  The GeoSOM has the advantage of fast neighborhood searching.
Increasing the size or number of neurons is done by tesselating the sides of
the icosahedron.  The number of neurons, N is thus '''N = f\^2*10+2'''.  As such control of the number of neurons is severly limited for large SOMs. A frequency of 30 yield 9002 nuerons and a frequency of 31 yeild 9612.
.

.

.

Agent based distributed computing will be employed in an attempt to reduce training time.  Observations will act as intelligent agents. Each agent will continually seek its best matching unit (BMU) and calculate the BMU's neighborhood given the current kernel width.  An observer will inform agents when and which neurons are changed as it happens.  Agents will communicate with the observer via sockets allowing them to be run from anywhere on a network.
\end{abstract}
\end{document}





