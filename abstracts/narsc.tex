\documentclass[11pt]{article}
\begin{document}
\title{PySOM: A Graph Based Implementation of Self-Organizing Maps}
\author{\sc{Charles Schmidt} - Geoda Center, Arizona State University\\
	\sc{Serge Rey} - Geoda Center, Arizona State University\\
	\sc{Andr\`e Skupin} - San Diego State University}

\date{August 1st, 2008}
\maketitle
\begin{abstract}
The traditional Self-Organizing Map (SOM) uses a rectangular or hexagonal
network topology.  These topologies create a well-known problem in SOM called
the boundary or edge effect.  Toroidal and spherical topologies have been
widely suggested as a solution to the problem.  However, both toroidal and
spherical topologies introduce new effects into the SOM which have yet to be
fully explored.  Further, there are a variety of methods for arranging nodes
in a spherical network topology.  To optimize for speed, existing SOM packages
use separate data-structures for each topology they implement.  The most
widely available implementation of SOM is Teuvo Kohonen's SOM\_PAK.  SOM\_PAK
implements both the traditional rectangular and hexagonal topologies.
However, at the time this writing, implementations which support alternative
topologies were not readily available.  To address this issue we introduce
PySOM.  PySOM is an open source implementation of SOM written in the Python
Programing Language. We represent the topology of the SOM as a graph. The
graph structure provides the necessary information to determine neuron
adjacency and construct neighborhoods.  Use a graph optimizes for flexibility
at the increased cost of neighbor searches. By abstracting the topology, PySOM
can train a SOM using any topology for which a graph structure can be created.

\end{abstract}
\end{document}





