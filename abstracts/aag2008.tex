\documentclass[11pt]{article}
\begin{document}
\title{Effects of Irregular Topology in Spherical Self-Organizing Maps}
\author{\sc{Charles Schmidt} - San Diego State University\\
	\sc{Serge Rey} - San Diego State University\\
	\sc{Andr\`e Skupin} - San Diego State University}

\date{April 18th, 2008}
\maketitle
\begin{abstract}
A regular network topology is one in which every node on the network has
exactly the same number of adjacent nodes. Any topology involving an edge is
irregular. Arranging our lattice on the surface of a sphere seems to be an
obvious way to overcome the edge. However, there exist only five arrangements
on the sphere which are completely regular; these are the five platonic solids
(Ritter, 1999; Harris et al., 2000).  Any other arrangement of neurons on the
surface of the sphere will result in an irregular topology, as not all neurons
will have the same number of neighbors. The classic method for minimizing this
irregularity is to generate the spherical lattice by tessellating the sides of
the icosahedron (Nishio et al., 2006). While this method will always result in
a highly regular spherical topology, the main drawback is that the number of
neurons in the network (the network size), N, grows exponentially as
tessellations are applied. That results in only very coarse control over
network size.  A top sphere which are completely regular; these are the five
platonic solids (Ritter, 1999; Harris et al., 2000).  Any other arrangement of
neurons on the surface of the sphere will result in an irregular topology, as
not all neurons will have the same number of neighbors. The classic method for
minimizing this irregularater control over network size. Toward that end, we
will develop and test new diagnostics to measure and visualize
topology-induced errors in SOM.
\end{abstract}
\end{document}





