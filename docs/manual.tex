Intro
    PySOM is an implementation of the Self-Organizing Map training algorithm written in the Python Programming Language.  What makes PySOM unique from other implementations is that the network topology is not hard wired, instead we represent the topology as a graph.  The NetworkX python library is used to manage the graph.  This document is intended to provide an overview of PySOM and help users get started with training their SOMs.  In it's current form some basic knowledge of the Python Programming language is required to use PySOM I hope to provide a more robust graphical user interface in the future, but for now PySOM can only be accessed programmitcally.  In the following sections you will learn how to setup your topology, organize your data, set your training parameters, and train your SOM.  Some useful functions and instructions for visualizing your SOM will also be provided, but most if this will need to be done without the help of PySOM.

    Library Organization
        pysom
            som.py
                Contains the training code
            data.py
                Classes for reading Observation Data
            util
            topo
                hex.py
                rook.py
            thesis
                Code directly related to thesis
    Topology
        Any valid undirected NetworkX can be used as the topology for your SOM.  The nodes of the graph will be treated as neurons.
        Several utility function are provided to help creat your graphs.
    Data
    Training
    Visualize
Installation
    Requirements
        PySOM requires a Python Version 2.4 or later.  Major version releases of Python (eg. from 2.0 to 3.0) are not backwards compatible and PySOM should not be expected to work with future major releases.
        In addition the following Python Libraries are required.
        Version numbers used for development are provided, but future version are expected to work.
        Numpy (1.0.4)
        NetworkX (0.35.1)
        To create spherical topologies you also need the STRI_PACK and SXYZ_VORONOI software packages, these require a fortran complier.
        To create Geoesic topologies you will need the Dome software package.
        To visualize the SOM you will need ESRI's ArcMap or another method for creating ShapeFiles.
    PySOM
        You can download PySOM from http://code.google.com/p/pysom-thesis/downloads/list
Topology
    Intro
        NetworkX Undirected Graph
    Rectangular
    Hexagon
    Geodesic
        Dome
        stri_pack
        parseDelaunay
    Spherical
        Rahkmonv
        stri_pack
        parseDelaunay
Data
    FileFormat
        Dims
        Obs
        Obs
        Obs
        ....
    ObsFile
        Complete
        Sparse
Training
    Parameters
        Dims
        maxN
        tSteps
        alpha0
    Phase I
    Phase II
Saving
Loading
Mapping

