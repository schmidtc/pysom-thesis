
\documentclass[nototal,handout]{beamer}
\mode<presentation>
{
  \usetheme{Madrid}
  \setbeamercovered{transparent}
}

\usepackage{verbatim}
\usepackage{fancyvrb}
\usepackage[english]{babel}
\usepackage[latin1]{inputenc}
\usepackage{times}
\usepackage{tikz}
\usepackage[T1]{fontenc}
\usepackage{graphicx} %sjr added
\graphicspath{{figures/}}
\usepackage{hyperref}

\author[Schmidt]{Charles R. Schmidt}
\institute[SDSU]{Department of Geography\\San Diego State University}
\title[Irregular Topology in Spherical-SOM]{Effects of Irregular Topology in Spherical Self-Organizing Maps}
\subtitle{Thesis presentation}
\date[Thesis]{September 15, 2008}

% Delete this, if you do not want the table of contents to pop up at
% the beginning of each subsection:
\AtBeginSubsection[]
{
  \begin{frame}<beamer>
    \frametitle{Outline}
    \tableofcontents[currentsection,currentsubsection]
  \end{frame}
}


% If you wish to uncover everything in a step-wise fashion, uncomment
% the following command: 
\beamerdefaultoverlayspecification{<+->}
\begin{document}
\begin{frame}
  \titlepage
\end{frame}
\begin{frame}
  \frametitle{Outline}
  \tableofcontents[pausesections]
  % You might wish to add the option [pausesections]
\end{frame}



\section{Background} 

\subsection{The Self-Organizing Map} 

\begin{frame}
	\frametitle{The Self-Organizing Map}
 
\begin{block}{What is the SOM}
 \begin{itemize}
 \item  Artificial Neural Network
 \item  Unsupervised Competitive Learning Process
 \item  High Dimension Input Data
 \item  Each Neuron Models of a Portion of the Input Space
 \end{itemize}
 \end{block} 
\begin{block}{Topology}
  \begin{center}
  \begin{figure}
  \includegraphics[width=0.60\linewidth]{topology.png}
  \end{figure}
  \end{center}
 \end{block} \end{frame} 

\begin{frame}
	\frametitle{The Self-Organizing Map}
 
\begin{block}{Applications}
 \begin{itemize}
 \item  Data Visualization
 \item  Data Reduction
 \item  Clustering
 \end{itemize}
 \end{block} 
\begin{block}{Training}
 \begin{itemize}
 \item  Randomize input vectors
 \item  Randomly Initialize the neurons
 \item  Loop Until Map Converges
 \begin{itemize}
 \item  Grab an Input Vector
 \item  Find the Best Matching Neuron and its Neighborhood
 \item  Modify the Weights of the Neurons to Make them More Similar to the Input Vector
 \end{itemize}
 \end{itemize}
 \end{block} \end{frame} 

\begin{frame}
	\frametitle{Training}
 
\begin{block}{Assignment and Updating}
  \begin{center}
  \begin{figure}
  \includegraphics[width=0.90\linewidth]{input.png}
  \end{figure}
  \end{center}
 \end{block} \end{frame} 

\begin{frame}
	\frametitle{Training}
 
\begin{block}{Iterations}
  \begin{center}
  \begin{figure}
  \includegraphics[width=0.90\linewidth]{somtrain.png}
  \end{figure}
  \end{center}
 \end{block} \end{frame} 

\begin{frame}
	\frametitle{SOM as Dimensionality Reduction}
  \begin{center}
  \begin{figure}
  \includegraphics[width=0.90\linewidth]{dimensionReduction.png}
  \end{figure}
  \end{center}
 \end{frame} 

\begin{frame}
	\frametitle{SOM as Clustering}
  \begin{center}
  \begin{figure}
  \includegraphics[width=0.90\linewidth]{clustering.png}
  \end{figure}
  \end{center}
 \end{frame} 


\section{} 

\begin{frame}
	\frametitle{SOM as Clustering}
  \begin{center}
  \begin{figure}
  \includegraphics[width=0.90\linewidth]{clustermap.png}
  \end{figure}
  \end{center}
 \end{frame} 

\begin{frame}
	\frametitle{SOM and GIScience}
  \begin{center}
  \begin{figure}
  \includegraphics[width=0.50\linewidth]{book.png}
  \end{figure}
  \end{center}
 \end{frame} 

\subsection{Edge Effects in SOM} 

\begin{frame}
	\frametitle{Edge Effects in SOM}
  \begin{center}
  \begin{figure}
  \includegraphics[width=0.60\linewidth]{states.png}
  \end{figure}
  \end{center}
 \end{frame} 

\begin{frame}
	\frametitle{Edge Effects in SOM}
  \begin{center}
  \begin{figure}
  \includegraphics[width=0.90\linewidth]{edge1.png}
  \end{figure}
  \end{center}
 \end{frame} 

\begin{frame}
	\frametitle{Edge Effect}
 
\begin{block}{Edge density}
 \begin{itemize}
 \item  Higher density of input vectors mapped to edge neurons
 \item  Higher internal variance for edge neurons
 \end{itemize}
 \end{block} 
\begin{block}{Siberia-Alaska Effect}
 \begin{itemize}
 \item  Close in geographic space, widely separated in world map
 \item  GIS: travel across the date line
 \end{itemize}
 \end{block} \end{frame} 

\begin{frame}
	\frametitle{Edge Effects in Spatial Analysis}
 
\begin{block}{Spatial Econometrics}
 \begin{itemize}
 \item  Power of tests for dependence
 \item  Properties of estimators
 \item  Spillover measurement
 \begin{itemize}
 \item  Global spatial multipliers
 \item  Local effects
 \end{itemize}
 \end{itemize}
 \end{block} 
\begin{block}{Point Patterns}
 \begin{itemize}
 \item  Inflated nearest neighbor distances
 \item  Edge corrections
 \end{itemize}
 \end{block} 
\begin{block}{Components of Edge Effect}
 \begin{itemize}
 \item  Buffering = Omission
 \item  True Boundary
 \end{itemize}
 \end{block} \end{frame} 

\begin{frame}
	\frametitle{Suggested Solutions}
 
\begin{block}{With Edges}
 \begin{itemize}
 \item  Hierarchical SOM
 \item  Growing SOM
 \item  Mathematical Weighting
 \end{itemize}
 \end{block} 
\begin{block}{Without Edges}
 \begin{itemize}
 \item  Spherical SOM
 \item  Torus SOM
 \end{itemize}
 \end{block} \end{frame} 

\begin{frame}
	\frametitle{Spherical SOM}
 
\begin{block}{Geodesic}
 \begin{itemize}
 \item  Highly Regular
 \item  Most Common
 \item  Limited Network Size
 \end{itemize}
 \end{block} 
\begin{block}{Rakhmanov}
 \begin{itemize}
 \item  Less Regular
 \item  Rejected in Literature
 \item  No Network Size Limitation
 \end{itemize}
 \end{block} \end{frame} 

\begin{frame}
	\frametitle{Spherical SOM}
  \begin{center}
  \begin{figure}
  \includegraphics[width=0.60\linewidth]{642.png}
  \end{figure}
  \end{center}
 \end{frame} 

\begin{frame}
	\frametitle{Tradeoffs}
 
\begin{block}{Regularity}
 \begin{itemize}
 \item  Connectivity
 \item  Shape
 \item  Area
 \end{itemize}
 \end{block} 
\begin{block}{Granularity of network size}
 \begin{itemize}
 \item  Controlling regularity \emph{and} network size
 \item  Comparative analysis not straightforward
 \end{itemize}
 \end{block} \end{frame} 

\subsection{Objectives} 

\begin{frame}
	\frametitle{Objectives}
 
\begin{block}{Question}
  Does Regularity Matter in Spherical SOM?
 \end{block} 
\begin{block}{Methods}
 \begin{itemize}
 \item  Three Diagnostics
 \begin{itemize}
 \item  Internal Heterogeneity vs. First-Order Neighborhood Size
 \item  Internal Heterogeneity vs. Topological Regularity
 \item  Visualization of Internal Heterogeneity
 \end{itemize}
 \end{itemize}
 \end{block} \end{frame} 


\section{Methodology} 

\subsection{Data Generation} 

\begin{frame}
	\frametitle{Data Generation}
 
\begin{block}{Synthetic Multivariate Data}
 \begin{itemize}
 \item  Follow Wu and Takatsuka (2006)
 \item  Seven clusters
 \item  Three dimension
 \item  Clusters are normally distributed w/ unit variance on each dimension
 \item  Sample size varies to ensure adequate assignments to neurons
 \end{itemize}
 \end{block} \end{frame} 

\subsection{Internal Heterogeneity} 

\begin{frame}
	\frametitle{Internal Heterogeneity}
 
\begin{block}{Neuron Internal Heterogeneity}
  \({IH_i} = \frac{2}{{n_i}^2-{n_i}}\sum_{j=1}^{n_i}\sum_{k=j+1}^{n_i} ||{x_{ij}}-{x_{ik}}||\)
 \end{block} 
\begin{block}{Internal Heterogeneity}
 \begin{itemize}
 \item  Measures the difference of observations mapped in neuron
 \item  Measure of how well data fits to neuron
 \end{itemize}
 \end{block} \end{frame} 

\subsection{Diagnostics} 

\begin{frame}
	\frametitle{Diagnostics}
 
\begin{block}{Internal Heterogeneity vs. First-Order Neighborhood Size}
 \begin{itemize}
 \item  Does a neuron's I.V. decrease as its first-order neighborhood size increases?
 \begin{itemize}
 \item  Group neurons by degree
 \item  Measure I.V. within each group
 \item  Check for difference of means and variance between groups
 \end{itemize}
 \item  Repeat for each Topology
 \end{itemize}
 \end{block} \end{frame} 

\begin{frame}
	\frametitle{Diagnostics}
 
\begin{block}{Internal Heterogeneity vs. Topological Regularity}
 \begin{itemize}
 \item  Does a SOM's average I.V. increase in more irregular topologies?
 \begin{itemize}
 \item  Group neurons by Topology
 \item  Measure I.V. within each group
 \item  Check for difference of means and variance between groups
 \end{itemize}
 \end{itemize}
 \end{block} \end{frame} 

\begin{frame}
	\frametitle{Diagnostics}
 
\begin{block}{Visualization of Internal Heterogeneity}
 \begin{itemize}
 \item  What information can be learned from a mapping of the I.V.?
 \end{itemize}
 \end{block} \end{frame} 


\section{Results} 

\subsection{Internal Heterogeneity vs. First-Order Neighborhood Size} 

\begin{frame}
	\frametitle{Internal Heterogeneity vs. First-Order Neighborhood Size}
 
\begin{block}{Rectangular}
  \begin{center}
  \begin{figure}
  \includegraphics[width=0.75\linewidth]{rook_iv_box.png}
  \end{figure}
  \end{center}
 \end{block} \end{frame} 

\begin{frame}
	\frametitle{Internal Heterogeneity vs. First-Order Neighborhood Size}
 
\begin{block}{Hexagonal}
  \begin{center}
  \begin{figure}
  \includegraphics[width=0.75\linewidth]{hex_iv_box.png}
  \end{figure}
  \end{center}
 \end{block} \end{frame} 

\begin{frame}
	\frametitle{Internal Heterogeneity vs. First-Order Neighborhood Size}
 
\begin{block}{Geodesic}
  \begin{center}
  \begin{figure}
  \includegraphics[width=0.75\linewidth]{geodesic_iv_box.png}
  \end{figure}
  \end{center}
 \end{block} \end{frame} 

\begin{frame}
	\frametitle{Internal Heterogeneity vs. First-Order Neighborhood Size}
 
\begin{block}{Spherical}
  \begin{center}
  \begin{figure}
  \includegraphics[width=0.75\linewidth]{graph_iv_box.png}
  \end{figure}
  \end{center}
 \end{block} \end{frame} 

\subsection{Internal Heterogeneity vs. Topological Regularity} 

\begin{frame}
	\frametitle{Internal Heterogeneity vs. Topological Regularity}
  \begin{table}
  \centering
  \begin{minipage}{\textwidth}
  \caption{Measure of Topological Regularity and Sample Mean and Variance}
  \label{vardeg}
  \begin{tabular}{|c||c|c|c|}
  \hline
  &$c=1/n \sum_{i=1} (1/n \sum_{j=1} d_{i,j})^{-1}$&&\\
  Topology & Closeness Centrality & Mean & Variance\\
  \hline
  Rectangular & 0.0603 & 0.0294 &0.0001\\
  Hexagonal & 0.0739 & 0.0289 &0.0001\\
  Geodesic & 0.0890 & 0.0281 &0.0001\\
  Spherical & 0.0906 & 0.0281 &0.0001\\
  \hline
  \end{tabular}
  \end{minipage}
  \end{table}
 \end{frame} 

\begin{frame}
	\frametitle{Internal Heterogeneity vs. Topological Regularity}
  \begin{table}
    \begin{minipage}{\textwidth}
    \caption{Results of Difference of Mean Testing Across Topologies}
    \label{rlt:all}
    \begin{tabular}{|c||c|c|c|c|}
    \hline
    \textbf{Topology}&Rectangular				&Hexagonal &Geodesic &Spherical\\\hline
    \hline
 		Rectangular & (0.000064) & \textbf{0.001000} & \textbf{0.000100} & \textbf{0.000100}\\\hline
 		Hexagonal & -0.000479 & (0.000064) & \textbf{0.000100} & \textbf{0.000100}\\\hline
 		Geodesic & -0.001329 & -0.000850 & (0.000060) & 0.505600\\\hline
 		Spherical & -0.001328 & -0.000849 & 0.000001 & (0.000061)\\\hline
 
    \end{tabular}
    \end{minipage}
  \end{table}
 \end{frame} 

\subsection{Visualization of Internal Heterogeneity} 

\begin{frame}
	\frametitle{Visualization of Internal Heterogeneity}
 
\begin{block}{Rectangular}
  \begin{center}
  \begin{figure}
  \includegraphics[width=0.70\linewidth]{rook_clusters.png}
  \end{figure}
  \end{center}
 \end{block} \end{frame} 

\begin{frame}
	\frametitle{Visualization of Internal Heterogeneity}
 
\begin{block}{Hexagonal}
  \begin{center}
  \begin{figure}
  \includegraphics[width=0.70\linewidth]{hex_clusters.png}
  \end{figure}
  \end{center}
 \end{block} \end{frame} 

\begin{frame}
	\frametitle{Visualization of Internal Heterogeneity}
 
\begin{block}{Geodesic}
  \begin{center}
  \begin{figure}
  \includegraphics[width=0.70\linewidth]{geodesic_clusters.png}
  \end{figure}
  \end{center}
 \end{block} \end{frame} 

\begin{frame}
	\frametitle{Visualization of Internal Heterogeneity}
 
\begin{block}{Spherical}
  \begin{center}
  \begin{figure}
  \includegraphics[width=0.70\linewidth]{sphere_clusters.png}
  \end{figure}
  \end{center}
 \end{block} \end{frame} 


\section{Conclusions} 

\subsection{Conclusions} 

\begin{frame}
	\frametitle{Key Findings}
 
\begin{block}{Internal Heterogeneity vs. First-Order Neighborhood Size}
 \begin{itemize}
 \item  Significant differences for hexagonal and rectangular topologies
 \item  Significant difference in spherical topology
 \begin{itemize}
 \item  Magnitude of difference very small compared with magnitudes in "flat" topologies
 \end{itemize}
 \item  No significant difference for geodesic
 \end{itemize}
 \end{block} 
\begin{block}{Internal Heterogeneity vs. Topological Regularity}
 \begin{itemize}
 \item  Geodesic and spherical show no significant differences
 \end{itemize}
 \end{block} 
\begin{block}{Visualization of Internal Heterogeneity}
 \begin{itemize}
 \item  Shows patterns of clustering
 \item  Provides insight into the self-organizing process
 \end{itemize}
 \end{block} \end{frame} 

\begin{frame}
	\frametitle{Limitations}
 
\begin{block}{Qualifications}
 \begin{itemize}
 \item  Only addressing how topology effects the SOM.
 \begin{itemize}
 \item  Does not address the benefits of having or removing the edge.
 \item  Is pushing outliers to the edges a bad thing?
 \end{itemize}
 \item  First-order neighborhood size is capturing larger network irregularities in the "flat" topologies
 \end{itemize}
 \end{block} \end{frame} 

\begin{frame}
	\frametitle{Future Directions}
 
\begin{block}{Extensions}
 \begin{itemize}
 \item  Rotate sphere-based SOMs to show "central" feature
 \item  Compare different SOMs on more than just topology
 \item  Test other topologies
 \begin{itemize}
 \item  Remove nodes from the existing topologies
 \end{itemize}
 \end{itemize}
 \end{block} \end{frame} 

\begin{frame}
	\frametitle{Clustering Properties}
 
\begin{block}{Global Structure}
 \begin{itemize}
 \item To what extent is the underlying clustering captured by the different topologies and sizes?
 \item  Are the same number of clusters found?
 \end{itemize}
 \end{block} 
\begin{block}{Local Structures}
 \begin{itemize}
 \item  Do different topologies generate clusters in similar ways?
 \item  Which clusters are found most often across the different topologies?
 \end{itemize}
 \end{block} \end{frame}
\end{document}
